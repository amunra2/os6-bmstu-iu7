\chapter{Выполнение лабораторной работы}

\section{Программа 1}

\subsection{Однопоточная}

\begin{center}
    \captionsetup{justification=raggedright,singlelinecheck=off}
    \lstinputlisting[label=lst:app1,caption=Программа 1 (однопоточная)]{../src/app1.c}
\end{center}


\subsection{Обоснование}

Программа считывает информацию из файла <<alphabet.txt>>, который содержит англиский алфавит -- строку символов\linebreak <<ABCDEFGHIJKLMNOPQRSTUVWXYZ>>. В результате своей работы программа  при помощи двух буферов посимвольно выводит считанные символы в стандартный поток вывода \textit{stdout}.

В функции \texttt{main} создается файловый дескриптор для открытого на чтение файла <<alphabet.txt>> с помощью функции \texttt{open()}. Затем при помощи \texttt{fdopen()} создаются два указателя на структуру \texttt{FILE}, в которой поле \texttt{\_fileno = 3} (дескриптор, который вернула функция \texttt{open()}).
Функция \texttt{setvbuf()} изменяет тип буферизации для \texttt{fs1} и \texttt{fs2} на полную буферизацию, а также явно задает размер буфера (\texttt{18} байт).

При первом вызове \texttt{fscanf()} буфер \texttt{buf1} будет заполнен полностью (все 13 символов). При этом значение \texttt{f\_pos} в структуре \texttt{struct\_file} открытого файла увеличится на \texttt{18}. Далее при первом вызове \texttt{fscanf()} для \texttt{fs2} в \texttt{buf2} считаются оставшиеся \texttt{8} символов, начиная с \texttt{f\_pos} (т.к. \texttt{fs1} и \texttt{fs2} ссылаются на один и тот же дескриптор \texttt{fd}).

После чего в цикле поочередно будут выведены символы из \texttt{buf1} и \texttt{buf2} (при этом после 8 итераций символы будут выводиться только из \texttt{buf1}).

\imgHeight{80mm}{app1_scheme}{Связь между дескрипторами в первой программе}

\imgHeight{30mm}{app1_result}{Результат работы первой программы (однопоточная)}


\subsection{Многопоточная}

\begin{center}
    \captionsetup{justification=raggedright,singlelinecheck=off}
    \lstinputlisting[label=lst:app1,caption=Программа 1 (многопоточная)]{../src/app1_threads.c}
\end{center}


\imgHeight{20mm}{app1_threads_result}{Результат работы первой программы (многопоточная)}



\section{Программа 2}

\subsection{Однопоточная}

\begin{center}
    \captionsetup{justification=raggedright,singlelinecheck=off}
    \lstinputlisting[label=lst:app2,caption=Программа 2 (однопоточная)]{../src/app2.c}
\end{center}


\subsection{Обоснование}

В этой программе создаются два дескриптора открытого файла при помощи функции \texttt{open()}. При этом в системной таблице открытых файлов создаются две новых записи. Затем в цикле поочередно считываются символы из файла и выводятся на экран (при этом на экран символы будут выводиться дважды, так как каждый \texttt{open()} создаст структуру \texttt{struct file}, каждая из которых будет иметь свой \texttt{f\_pos}, поэтому смещения в файловых дескрипторах будут независимы).

\imgHeight{60mm}{app2_scheme}{Связь между дескрипторами во второй программе}

\imgHeight{20mm}{app2_result}{Результат работы второй программы (однопоточная)}


\subsection{Многопоточная}

\begin{center}
    \captionsetup{justification=raggedright,singlelinecheck=off}
    \lstinputlisting[label=lst:app2,caption=Программа 2 (многопоточная)]{../src/app2_threads.c}
\end{center}


\subsection{Обоснование}

При многопоточной реализации возникает та же проблема, но при этом порядок вывода элементов хаотичен. В качестве решения данной проблемы при многопоточной реализации был использован \texttt{mutex}, а также разделяемая область памяти, то есть переменная \texttt{filePos}, которая отслеживает положение указателя и увеличивает его на один при чтении элемента из файла (следующий элемент затем читается из файла с места \texttt{filePos} с использованием функции \texttt{lseek()}).


\imgHeight{20mm}{app2_threads_result}{Результат работы второй программы (многопоточная)}



\section{Программа 3}

\subsection{Однопоточная}

\begin{center}
    \captionsetup{justification=raggedright,singlelinecheck=off}
    \lstinputlisting[label=lst:app3,caption=Программа 3 (однопоточная)]{../src/app3.c}
\end{center}


\subsection{Обоснование}

В этой программе файл открывается 2 раза для записи. Ввод при этом выполняется через функцию буферизированного вывода \texttt{fprintf()} стандартной библиотеки \texttt{stdio.h} языка \texttt{С} (буфер создается без явного вмешательства). Изначально информация записывается в буфер, а затем переписывается в файл, если буфер полон, произошла принудительная запись функцией \texttt{fflush()} или если вызван \texttt{fclose}.

В программе буквы алфавита, которые имеют нечетный код в таблице \texttt{ASCII} записываются в буфер дескриптора \texttt{f1}, а четные --- в буфер дескриптора \texttt{f2}. Информация из буфера будет записана в файл при вызове \texttt{fclose()}. Но поскольку у каждого из дескрипторов свое поле \texttt{f\_pos}, то запись будет производиться с начала файла при вызове \texttt{fclose()} для \texttt{f1} и \texttt{f2}. В результате инфорация будет перезаписана при втором вызове \texttt{flcose()}.

Причем если сначала был вызван \texttt{flcose(f1)}, а затем \texttt{flcose(f2)}, то в файл будут записаны \textit{четные} буквы английского алфавита (рис. \ref{img:app3_file_reverse}), а если \texttt{flcose()} будут вызваны в обратном порядке, то в файл запишутся \textit{нечетные} буквы (рис. \ref{img:app3_file_normal}).


\imgHeight{8mm}{app3_file_reverse}{Результат при вызове fclose() для f1, затем f2}
\imgHeight{10mm}{app3_file_normal}{Результат при вызове fclose() для f2, затем f1}
\imgHeight{70mm}{app3_result_info}{Результат работы третьей программы - информация (однопоточная)}


\imgHeight{70mm}{app3_scheme}{Связь между дескрипторами в третьей программе}

\subsection{Многопоточная}

\begin{center}
    \captionsetup{justification=raggedright,singlelinecheck=off}
    \lstinputlisting[label=lst:app3,caption=Программа 3 (многопоточная)]{../src/app3_threads.c}
\end{center}


\subsection{Обоснование}

При многопоточной реализации возникает та же проблема в перезаписью. В данной реализации в качестве решения проблемы предложено открывать файл в режиме \texttt{a (append)}. При этом информация не будет перезаписана, а буферы будут в порядке вызова \texttt{fclose()} записаны в файл.


\imgHeight{8mm}{app3_threads_file}{Результат работы третьей программы (многопоточная)}


\chapter*{Вывод}

При работе с буферезированным вводом-выводом может возникнуть ряд проблем, которые демонстритруются в программах данной лабораторной работы. 

Во второй программе возникает проблема двойного вывод символов на экран, так как из-за двух файловых дескрипторов смещения в файле будут производиться независимо. Решением данной проблемы является использование \texttt{mutex} и создание разделяемой области памяти (некой переменной, которая будет отслеживать позицию указателя в файле).

В третьей программе демонстритруется проблема перезаписи информации при вызове \texttt{flcose()} для разных файловых потоков, так как каждый начинает запись с начала файла. Решением проблемы является открытие файла на дозапись в режиме \texttt{a (append)}.
